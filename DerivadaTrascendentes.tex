\subsection{Funciones trascendentes}

Usaremos indistintamente $\cot$ y $ctg$. Además, supondremos que cada función trigonométrica es continua en $v(x)$ como ya se había mencionado.
\subsubsection{Trigonométricas}
\Large
\begin{empheq}[box=\fbox]{align}
	\frac{d}{dx}\sen v(x)&=\cos v(x)\frac{d}{dx}v(x)\label{eqn:2.1}\\
	\frac{d}{dx}\cos v(x)&=-\sen v(x)\frac{d}{dx}v(x)\label{eqn:2.2}\\
	\frac{d}{dx}\tan v(x)&=[\sec v(x)]^2\frac{d}{dx}v(x)\label{eqn:2.3}\\
	\frac{d}{dx}\cot v(x)&=-[\csc v(x)]^2\frac{d}{dx}v(x)\label{eqn:2.4}\\
	\frac{d}{dx}\sec v(x)&=\sec v(x)\tan v(x)\frac{d}{dx}v(x)\label{eqn:2.5}\\
	\frac{d}{dx}\csc v(x)&=-\csc v(x)\cot v(x)\frac{d}{dx}v(x)\label{eqn:2.6}
\end{empheq}
\normalsize

\subsubsection{Inversas Trigonométricas}
\Large
\begin{empheq}[box=\fbox]{align}
	\frac{d}{dx}\arcsen v(x)&=\frac{1}{\sqrt{1-[v(x)]^2}}\frac{d}{dx}v(x)\label{eqn:2.7}\\
	\frac{d}{dx}\arccos v(x)&=-\frac{1}{\sqrt{1-[v(x)]^2}}\frac{d}{dx}v(x)\label{eqn:2.8}\\
	\frac{d}{dx}\arctan v(x)&=\frac{1}{1+[v(x)]^2}\frac{d}{dx}v(x)\label{eqn:2.9}\\
	\frac{d}{dx}\arccot v(x)&=-\frac{1}{1+[v(x)]^2}\frac{d}{dx}v(x)\label{eqn:2.10}\\
	\frac{d}{dx}\arcsec v(x)&=\frac{1}{v(x)\sqrt{[v(x)]^2-1}}\frac{d}{dx}v(x)\label{eqn:2.11}\\
	\frac{d}{dx}\arccsc v(x)&=-\frac{1}{v(x)\sqrt{[v(x)]^2-1}}\frac{d}{dx}v(x)\label{eqn:2.12}
\end{empheq}
\normalsize

\subsubsection{Ejercicios}
\begin{enumerate}
\item Demostrar las fórmulas \ref{eqn:2.7} a \ref{eqn:2.12} para el caso $v(x)=x$.
\item Usando la regla de la cadena y el ejercicio anterior, demostrar las fórmulas \ref{eqn:2.7} a \ref{eqn:2.12}.
\end{enumerate}

\subsubsection{Logarítmicas}
\Large
\begin{empheq}[box=\fbox]{align}
	\frac{d}{dx}\ln v(x)&=\frac{1}{v(x)}\frac{d}{dx}v(x)\label{eqn:2.13}\\
	\frac{d}{dx}\log_b v(x)&=\frac{\log_b e}{v(x)}\frac{d}{dx}v(x)\label{eqn:2.14}
\end{empheq}
\normalsize

\subsubsection{Exponenciales}
\large
\begin{empheq}[box=\fbox]{align}
	\frac{d}{dx}e^{v(x)}&=e^{v(x)}\frac{d}{dx}v(x)\label{eqn:2.15}\\
	\frac{d}{dx}a^{v(x)}&=\ln a\cdot a^{v(x)}\frac{d}{dx}v(x)\label{eqn:2.16}\\
	\frac{d}{dx}[u(x)]^{v(x)}&=v(x)\cdot[u(x)]^{v(x)-1}\frac{d}{dx}u(x)+\ln u(x)\cdot [u(x)]^{v(x)}\frac{d}{dx}v(x)\label{eqn:2.17}
\end{empheq}
\normalsize

\subsubsection{Ejercicios}

\begin{enumerate}
	\item Demostrar \cref{eqn:2.15} \textsl{[Hint: Usar la regla de la cadena.]}
        \item Determinar $\frac{d}{dx}\frac{\sen(2x)}{e^x}$
	\item Demostrar \cref{eqn:2.13} usando el ejercicio anterior y la regla de la cadena.
	\item Demostrar \cref{eqn:2.14} \textsl{[Hint: Usar la fórmula de cambio de base:]
		\begin{align*}
		\log_b y&=\frac{\ln y}{\ln b}\\
		&=\ln y\log_b e
		\end{align*}}
\end{enumerate}

Algunas de las fórmulas anteriores se demuestran, a grandes rasgos, de la siguiente forma:

\begin{itemize}
	\item Para demostrar \cref{eqn:2.16} se usa que $a^{v}=e^{\ln a\: v}$ y \cref{eqn:2.15}.
	\item Para demostrar \cref{eqn:2.17} se usa que $u^{v}=e^{\ln(u)\:\cdot v}$, además de \cref{eqn:2.15,eqn:1.6,eqn:2.13}.
\end{itemize}